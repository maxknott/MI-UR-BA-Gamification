% durch Austauschen dieser Zeilen kann die Sprache des Templates geändert werden
\PassOptionsToPackage{main=ngerman}{babel}
%\PassOptionsToPackage{main=english}{babel}

% durch Austauschen dieser Zeilen kann zwischen Abschlussarbeit und Seminararbeit gewechselt werden
\documentclass[thesis]{mi-document}
%\documentclass[seminar]{mi-document}

\bachelor % im Falle einer Masterarbeit \master
%\master

% Variablen, die für das Deckblatt und Metadaten verwendet werden
\title{Prototypische Implementierung und Evaluation von Game-Design-Elementen (Gamification) in einer mobilen Anwendung für transdiagnostische Psychoedukation}
\author{Maximilian Knott}
\semester{SS 2024}



\studid{2025274}
\studSemester{12. Semester B.A. Medieninformatik / Medienwissenschaft}
\phone{+49 15227464590} % Optional
\studSubject{Medieninformatik}
\firstReviewer{Prof. Dr. Christian Wolff}
\secondReviewer{Prof. Dr. Youssef Shiban}
\advisor{Vitus Maierhöfer}
\address{Am Bodenacker 8, 93138 Lappersdorf}{} % Optional
\mail{mxknott@gmail.com}
\studMail{maximilian.knott@stud.uni-regensburg.de}
\dateHandedIn{30.09.2024}
\keywords{UI, UX, Gamification, Mental Health Apps}


\bibliographystyle{apacite}

% Falls Sie die Abkürzung zum Einbinden von Grafiken benutzen möchten. Erläuterung fnden Sie im Abschnitt zu Abbildungen.
\input{config}

\begin{document}

% auskommentieren, damit Sachen nach Kapitel nummeriert werden (z.B. Abbildung 3.2)
\counterwithout{footnote}{chapter}
\counterwithout{figure}{chapter}
\counterwithout{table}{chapter}
\counterwithout{lstlisting}{chapter}

% Die Nummerierung beginnt mit der Titelseite (= Seite 1), soll aber erst ab der ersten Inhaltsseite (Einleitung) angezeigt werden.
\pagestyle{empty}

% Deckblatt des Templates und Hinweise
% diese Zeile für die Verwendung des Templates entfernen!
%\input{hinweise}                                            TODO: delete file hinweise.tex

% Das Deckblatt erstellen
\maketitle

\tableofcontents % Optional
\listoffigures % Optional
\listoftables % Optional                                     TODO: probably not needed so delete later
\lstlistoflistings % Optional


\clearpage
\doublespacing

% Abstract hier
\input{abstract}

\clearpage
\pagestyle{headings} % Seitennummern und Kapitelbezeichnungen anzeigen

% hier beginnt der eigentliche Inhalt der Arbeit


%\include{aufgabenstellung}                                 TODO: delete file aufgabenstellung.tex
\include{einleitung}
%\include{ziele}                                            TODO: delete file ziele.tex
\chapter{Theoretischer Hintergrund}\label{sec:theoretischer_hintergrund}

%TODO: hier noch Einleitungssatz einfügen

\section{Psychische Gesundheit in der Bevölkerung}\label{subsec:psychische_gesundheit}

Das Thema mentale Gesundheit ist in den vergangenen Jahren in der Gesellschaft immer präsenter geworden. Dies ist unter anderem auch auf eine zunehmende mediale Berichterstattung mit einer offeneren Darstellung von psychischen Krankheiten zurückzuführen, die nicht zuletzt auch durch ein generell gestiegenes gesellschaftliches Bewusstsein und eine Sensibilisierung der Menschen bezüglich mentaler Gesundheit und psychischer Störungen bedingt ist.

\subsection{Entwicklung psychischer Gesundheit der Bevölkerung}\label{subsubsec:entwicklung_psychische_gesundheit}

In routinemäßig gesammelten administrativen Daten von Krankenkassen und Rentenversicherungen, 
sowie in Zahlen über durch psychische Beschwerden bedingte Arbeitsausfälle 
lässt sich ein Anstieg der Fälle verzeichnen 
\cite[]{latex:arzteblatt,latex:dak}. \\ %(Ärzteblatt, 2024; DAK-Gesundheit 2024).
In der Publikation “Psychreport 2024”, einer jährlich erscheinenden Datenaufbereitung, 
die von der deutschen Krankenkasse DAK-Gesundheit veröffentlicht wird, 
werden verschiedene relevante Daten im Bezug auf psychische Gesundheit 
und deren Entwicklung mit besonderem Fokus auf den Arbeitskontext 
in verschiedenen Betrachtungszeiträumen gesammelt dargestellt. \\

Demnach stieg die Anzahl der krankheitsbedingten Arbeitsausfälle 
aufgrund von psychischen Erkrankungen im Jahr 2023 verglichen mit dem Vorjahr um 21\% 
und liegt mittlerweile hinter Erkrankungen des Atmungssystems 
und Muskel-Skelett-System Erkrankungen auf Platz drei der Erkrankungsgruppen, 
die die meisten Ausfalltage bei der Arbeit verursachen. 

Ein besonders starker Anstieg der Arbeitsunfähigkeits-Fälle ist bei der Altersgruppe der 20-24 Jährigen 
sowie der 25-29 Jährigen zu erkennen. Hier stiegen die Fallzahlen 2023 
um 34\% bzw. 31\% im Vergleich zum Vorjahr. \\
In den 10 Jahren im Zeitraum 2013-2023 ist insgesamt ein Anstieg 
der durch psychische Erkrankungen bedingten Fehltage bei der Arbeit von 52\% festzustellen 
\cite[]{latex:dak}. %(DAK Gesundheit, 2024).

Ein weiterer Indikator für den Zustand der psychischen Gesundheit der Bevölkerung ist die Diagnoseprävalenz, 
bei der administrative Daten betrachtet werden, 
um unter anderem die Auswirkungen von Diagnosen psychischer Störungen auf die ärztliche und psychotherapeutische Behandlung zu analysieren \cite[]{thom2024trends}. \\ %(Thom, et al., 2024).

Allerdings sind diese Daten nur eingeschränkt als Indikator für die Krankheitshäufigkeit in der Bevölkerung tauglich. 
Gründe hierfür können unter anderem darauf zurückgeführt werden, 
dass manche Kranke keine ärztliche oder psychotherapeutische Hilfe in Anspruch nehmen 
und dass Behandelnde psychische Störungen in machen Fällen entweder gar nicht erfassen 
oder sogar überdiagnostizieren \cite[]{thom2024trends}. \\ %(Thom, et al. 2024). 
Die Diagnoseprävalenz-Zahlen verändern sich unterschiedlich stark über verschiedene Betrachtungszeiträume und durch das Heranziehen verschiedener Auswertungen mit unterschiedlicher Methodik und Thematik lassen sich daraus schwer konkrete Schlussfolgerungen ziehen. Trotzdem lässt sich in den Diagnoseprävalenz-Daten zu psychischen Störungen im Betrachtungszeitraum 2012-2022 ein relativer Zuwachs von über 13\% beobachten \cite[]{thom2024trends}. %(Thom, et al., 2024). 

Eine durch das Robert-Koch-Institut beobachtete Zunahme von Symptomen zu Depressionen und Angststörungen im Zeitraum 2019-2024 \cite[]{latex:rki} %(RKI, 2024) 
scheint sich allerdings zumindest bis 2022 nicht in einer Zunahme der ambulanten Diagnoseprävalenz niederzuschlagen \cite[]{thom2024trends}. %(Thom, et al., 2024). 


Auch wenn aufgrund dieser Trends in den administrativen Daten bei manchen zunächst die instinktive Reaktion ausgelöst werden könnte, alarmierende Schlussfolgerungen daraus zu ziehen, ist bei der Interpretation der Zahlen Vorsicht geboten.
Manche Medien nutzen diese Daten beispielsweise, um eine generelle Zunahme von psychischen Störungen in der Bevölkerung zu suggerieren \cite[]{latex:nickels,latex:knollenborg}. %(Nickels, 2020; Knollenborg, 2024).
Vieles deutet darauf hin, dass für eine verlässliche Interpretation der beobachteten Trends ein umfassende Auswertung kontextualisierender Evidenz nötig ist, was allerdings durch einen Mangel an geeigneten epidemiologischen Studien für den Beobachtungszeitraum in Deutschland erschwert wird 
\cite[]{thom2024trends}. %(Thom, et al., 2024).


Es ist also kritisch zu betrachten, ob die zunehmende Präsenz von psychischen Störungen sowohl in der Öffentlichkeit, als auch in den administrativen Daten, auch als Hinweis auf eine gestiegene Zahl von tatsächlich psychisch kranken Menschen interpretiert werden kann \cite[]{wagner2021psychische}. %(Wagner, 2021).
Epidemiologischen Studien zufolge kann ein solcher Anstieg an psychischen Störungen nicht unbedingt bestätigt werden. Auch eine Diskussion über mögliche Ursachen eines vermeintlichen Anstiegs psychischer Störungen sollte kritisch betrachtet werden.
Manche Veröffentlichungen leiten die Ursachen von gesellschaftlichen Faktoren wie z.B. einer “krankmachenden Beschleunigung der Arbeitswelt durch Digitalisierung und neue Medien” ab 
\cite[S. 6]{wagner2021psychische}. %(Wagner, 2021, S.6). 
Einerseits ist es berechtigt, in diesem Kontext gesellschaftliche Faktoren heranzuziehen, weil diese durchaus auch krank machen und Leiden verursachen können. Zudem gibt es im Feld der Psychiatrie, verglichen mit anderen medizinischen Bereichen, die Besonderheit, dass eine Krankheit hier besonders eng mit dem sozialen und gesellschaftlichen Kontext verbunden ist und deshalb schwer abgekoppelt von der gesellschaftlichen Wahrnehmung betrachtet werden kann. “Andererseits bestimmen gesellschaftliche Diskurse auch, wie Unbehagen, Unzufriedenheit, subjektives Leid von den Betroffenen und ihren Angehörigen gedeutet wird” \cite[S. 6]{wagner2021psychische}. %(Wagner, 2021, S.6). 
Also kann das übermäßige Thematisieren von einer Zunahme psychischer Störungen auch dazu führen, dass normal vorkommende nicht vermeidbare menschliche Leidenszustände (wie z.B. länger als zwei Wochen andauernde Trauer nach dem Tod von Angehörigen) oder vereinzelt auftretende Beeinträchtigungen zu schnell pathologisiert und medikalisiert werden. Dies kann unter anderem zur Folge haben, dass “Gesunde beunruhigt und therapiert werden, während für wirklich Kranke die Ressourcen fehlen” 
\cite[S. 11]{wagner2021psychische}. %(Wagner, 2021, S. 11).


\subsection{Folgen von psychischen Krankheiten}\label{subsubsec:folgen_psychischer_krankheiten}


Psychische Störungen können neben systemischen Auswirkungen, wie der wirtschaftlichen Belastung für die Gesellschaft, auch schwere persönliche Folgen für Betroffene und ihre Angehörigen haben. Depressionen und Angststörungen beeinträchtigen weite Teile des Lebens, von der Leistungsfähigkeit bei der Arbeit bis zum sozialen Bereich und der Gesundheit \cite[]{brenes2007anxiety} %(Brenes, 2007). 
Bei depressiven Erwachsenen ist die Lebensqualität stärker beeinträchtigt, als bei Erwachsenen mit Krankheiten wie Diabetes, erhöhtem Blutdruck oder chronischen Lungenerkrankungen und von Angststörungen Betroffene sind durch verminderte Arbeitsproduktivität, eingeschränkte soziale Kontaktfähigkeit, physischen Gebrechen und sogar Sterblichkeit betroffen \cite[]{weisel2021digital,brenes2007anxiety}  %(Weisel, 2021; Brenes, 2007). 
Wenn bei physischen Krankheiten wie der Koronaren Herzerkrankung psychische Störungen wie Depressionen und Angststörungen als Komorbidität (Begleiterkrankung) auftreten, kann das zu einem erhöhten Sterblichkeitsrisiko beitragen (Doering et al., 2010). 
Abhilfe für die durch psychische Störungen ausgelösten Beschwerden kann neben einer medikamentösen Behandlung durch Psychopharmaka vor allem eine Psychotherapie schaffen. Die Wirksamkeit von verschiedenen psychotherapeutischen Interventionen ist weitreichend zweifelsfrei belegt. Durch Psychotherapie profitieren durchschnittlich 50-70\% der Behandelten, wobei eine schnellere, stärkere und nachhaltigere Wirkung erzielt werden kann, als durch einen natürlichen Heilungsprozess oder ein unterstützendes Umfeld allein (Wagner, 2021 S.55).



\subsection{Versorgungsdefizite des Gesundheitssystems}\label{subsubsec:versorgungsdefizite}



\section{Digitale psychische Gesundheitspflege}\label{subsec:digitale_psychische_gesundheit}


\subsection{interaktive selbsthilfe-Übungen}\label{subsubsec:interaktive_selbsthilfe_uebungen}

\subsection{Mental Health Apps}\label{subsubsec:mental_health_apps}



\section{Gamification in Mental Health Apps}\label{subsec:gamification_in_mental_health_apps}


\subsection{Definition von Gamification}\label{subsubsec:definition_gamification}

\subsection{Einfluss von Gamification auf Motivation und Leistung}\label{subsubsec:einfluss_gamification_motivation_leistung}



\section{Forschungsprojekt}\label{subsec:forschungsprojekt}

%citation in text with page number
Das ist eine Zitierung im Fließtext \cite[S. 6]{wagner2021psychische} oder \cite[S. 11]{mustermann2013test}. Zum austesten.


Es gibt zahlreiche Ratgeber\index{Ratgeber} für das wissenschaftliche Arbeiten und Schreiben. Die Handbücher unterscheiden sich in inhaltlichen Schwerpunkt, praktischer Orientierung und Vertiefung der einzelnen Themen. Drei sehr empfehlenswerte Ratgeber sollen kurz vorgestellt werden.

\cite{karmasin2012gestaltung} bieten einen sehr knappen und praktisch orientierten Ratgeber. Es werden inhaltliche und formale Anforderungen an wissenschaftliche Arbeiten wie inhaltlicher Aufbau der Kapitel, Bewertungskriterien und formale Aspekte wie Gliederung behandelt. Daneben enthält der Ratgeber ein eigenes Kapitel mit Tipps zur Formatierung mit Word.

Das Handbuch von \cite{esselborn2012richtig} konzentriert sich auf die Frage nach dem richtigen wissenschaftlichen Sprachstil. Es werden konkrete Regeln und Übungen vorgestellt um sprachliche Präzision und gedankliche Klarheit im Text zu erreichen. Daneben wird in einem eigenen Kapitel auf die häufigsten Fehler beim wissenschaftlichen Schreiben hingewiesen.

\cite{balzert2011wissenschaftliches} bieten einen sehr ausführlichen Ratgeber zum wissenschaftlichen Arbeiten. Im ersten Teil werden Qualitätskriterien und Methoden als Grundlagen wissenschaftlicher Arbeit aufgezeigt. Im zweiten Teil werden verschiedene wissenschaftliche Artefakte also Textformen gegenübergestellt und der formale Aufbau wissenschaftlicher Arbeiten beleuchtet. Im dritten Teil werden Empfehlungen zum Erstellungsprozess einer Arbeit mit Projektplan etc. gegeben. Im letzten Teil werden verschiedene Aspekte der Präsentation behandelt, wie z. B. Vortragsformen mit und ohne visuelle Unterstützung oder der richtige Vortragsstil.

\include{gestaltungsrichtlinien}
\include{empfehlungen}
\include{zusammenfassung}


% Kapitelbezeichnung in der rechten oberen Ecke entfernen
\clearpage
\pagestyle{plain}

% Literaturverzeichnis anzeigen
% kleinerer Zeilenabstand, damit es nicht so gestreckt aussieht
\onehalfspacing
\bibliography{literature}                                   %TODO: switch to literature_new.bib.. or not
\doublespacing

% Anhang                                                    TODO: probably remove anhang but not sure
\appendix
\include{anhang}

% Tipps zur Verwendung von LaTeX                            TODO: remove latex tips
%\include{latex}

\begin{singlespace}
\KOMAoptions{parskip=full}
% Erklärung zur Urherberschaft (urheberschaft.tex) anhängen
\include{urheberschaft}

% Erklärung zur Lizenzierung der Arbeit (lizenzierung.tex) anhängen
\include{lizenzierung}

% Stichwortverzeichnis anzeigen. Weiß nicht, warum das nicht nach dem Inhaltsverzeichnis kommt.
\printindex
\end{singlespace}

% Inhalt des Datenträgers. Gehört meiner Meinung nach zum Anhang, aber was weiß ich schon. AS
\newpage
\input{datenträger}

\end{document}
